\documentclass[a4paper,12pt]{report}
\usepackage[utf8]{inputenc}
%\usepackage[frenchb]{babel}
\usepackage[T1]{fontenc}
\usepackage{lmodern,textcomp}
\usepackage{graphicx}
\usepackage{listings}
\usepackage{caption}
\usepackage{fancybox}
\usepackage[pdftex]{hyperref}
\usepackage[usenames,dvipsnames]{pstricks}
\usepackage{epsfig}
\usepackage{fancyvrb}
\usepackage{tikz}
\usetikzlibrary{shapes.geometric,backgrounds,fit,positioning,trees}
\usepackage{alltt}
\usepackage{wrapfig}
\usepackage{xcolor}

\definecolor{reddebian}{rgb}{0.84314,0.03922,0.32549}
\definecolor{bluedane}{rgb}{0.09020,0.56863,1}
\definecolor{greendane}{rgb}{0.43137,0.60784,0.14510}


\hypersetup{%
  colorlinks= true,
  linkcolor = greendane,
  urlcolor = bluedane
  }
\hypersetup{
  pdftitle={Xia},
  pdfauthor={Énuma Logiciel Libre},
  pdfsubject={Xia},
  pdfkeywords={Xia, logiciel libre, html5, Inkscape}
}



\renewcommand{\thechapter}{\arabic{chapter}}
\renewcommand{\thesection}{\Roman{section}}
\renewcommand{\thesubsection}{\alph{subsection}}

% pour unifier les indications relatives aux manipulation à effectuer dans les logiciels
% à modifier au besoin
\newcommand{\chemin}[1]{\texttt{\textcolor{reddebian}{#1}}}
